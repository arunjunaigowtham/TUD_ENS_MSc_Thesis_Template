\chapter{Introduction}
\label{chp:introduction}
In today's interconnected world, the advancement in Internet of Things (IoT) devices is reshaping the way we interact with technology. These devices have become indispensable tools, enabling smart homes, industrial automation, healthcare monitoring, and more. As the demand for IoT devices continues to rise, so does the need for sustainable power solutions to ensure their seamless operation. About 78 million batteries powering IoT devices will be dumped globally every day by 2025 if nothing is done to improve their lifespan \cite{2023up}. This dire statistic comes from EnABLES, an EU-funded project that’s urging researchers and technologists to take action to ensure that batteries outlive the devices they power \cite{2023up}.
\vspace{1\baselineskip}

\noindent Battery Free IoT devices have emerged as a promising solution to address the challenges posed by finite battery life. These devices tap into ambient energy sources or energy harvesting techniques to power their operations, offering a sustainable and maintenance-free approach. This paradigm shift holds the potential to revolutionise various domains, ranging from healthcare to environmental monitoring, by enabling devices to function indefinitely without battery replacement.
\vspace{1\baselineskip}

\noindent Wireless sensor networks, a cornerstone of IoT infrastructure, stand to gain immensely from the advent of battery-less IoT devices. These networks, composed of interconnected sensors, offer unparalleled data collection and monitoring capabilities \cite{2021Battery}. However, the reliance on traditional battery-powered sensors often limits their deployment due to the need for frequent maintenance and replacement. The integration of battery free IoT devices into wireless sensor networks presents a game-changing opportunity, promising extended operational lifetimes and reduced maintenance overhead.
\vspace{1\baselineskip}

\noindent Within this landscape, The FreeBie technology, which provides Bluetooth Low Energy (BLE) communication on intermittent battery-free IoT devices, has emerged as a trailblazing solution. FreeBie is the first battery-free active wireless system that sustains bi-directional communication on intermittently harvested energy. With this, FreeBie opens doors to novel applications and possibilities for battery-less IoT devices \cite{de2022Intermittently}.

\section{Problem Statement}
Despite the immense potential it holds, a significant constraint is that FreeBie only supports intermittent-to-continuous device connections. That is in FreeBie, while the end device is intermittent-powered and battery-free, the BLE hosts to which FreeBie connects still requires continuous power. The central challenge revolves around achieving synchronisation within a fully battery-free architecture. This synchronisation is crucial for enabling the transition from connections with continuously powered hosts to connections with intermittently powered hosts.

\section{Research Goals}
Embracing this challenge, the core focus of this research is on devising a synchronisation mechanism that utilises a shared external signal. This method should seamlessly integrate into the existing FreeBie battery-free infrastructure, enabling intermittent-to-intermittent devices to synchronise and establish connections effectively.
\vspace{1\baselineskip}

\noindent In this thesis, we also introduce a novel concept of using human heart pulses as a shared external signal. This approach, while limited to human body wireless sensor networks, offers substantial benefits within healthcare applications.
\vspace{1\baselineskip}

\noindent In order to achieve this core objective, we outline three key contributions which collectively address the previously stated problem within the FreeBie architecture:

\begin{itemize}
    \item We design a novel synchronisation algorithm capable of effectively detecting heart rate-based pulses and establish BLE connection.
    
    \item We integrate the devised framework into the FreeBie architecture as a supplementary enhancement. This enable Bluetooth end devices and hosts to synchronise using heart rate, significantly reducing connection time and energy.
    
    \item Validate the proposed system's effectiveness on intermittently-powered end devices and compare its performance to naive and state of the art systems.
\end{itemize}
\vspace{8\baselineskip}

\section{Thesis Structure}
The structure of this report is as follows. Chapter \ref{chap:literature} presents related literature and explore them within the domain of battery-less IoT devices and wireless body sensor networks. Building upon the foundation laid out, Chapter \ref{chap:architecture} outlines the architecture and system design of the integrated solution. The subsequent Chapter \ref{chap:implementation} details the technical implementation of the proposed methodology. Chapter \ref{chap:results} showcases the results obtained from experimental evaluations and performance analyses.  Chapter \ref{chp:futurework} highlights potential future research avenues that could lead to further advancements for FreeBie and the proposed system. Finally, Chapter \ref{chp:conclusions} concludes this thesis by summarising contributions, and discussing the potential applications of the devised framework.