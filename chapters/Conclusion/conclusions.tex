\chapter{Conclusions}
\label{chp:conclusions}

In this thesis, we have introduced the CardioSync framework, a novel solution that addresses a significant limitation in the state-of-the-art intermittent BLE device, FreeBie. The core challenge of achieving true intermittency between two end nodes and forming a reliable BLE network has been successfully overcome through CardioSync. This framework harnesses the capabilities of the low-power heart rate monitoring sensor, to synchronise connection setups between two battery-free nodes, employing the human heart rate as a shared clock pulse. By leveraging the inherent rhythm of the human heart, CardioSync achieves an average connection time that is nearly \textbf{1.8 times faster} compared to the FreeBie model's asynchronous periodic connection setup strategy. As promising as this is, it comes with an understandable trade-off of increased power demand for the connection setup due to sensor integration. The CardioSync framework demands \textbf{3.70 times more power} than reference FreeBie model.
\vspace{1\baselineskip}

\noindent This novel approach paves the way for a wide range of applications, especially in the field of Body Sensor Networks (BSN). The ability to establish connections between battery-free devices without relying on external power sources holds immense promise for healthcare monitoring, environmental sensing, and beyond. Looking ahead, the implications of the CardioSync framework on wearable health monitoring and IoT are significant. Its success opens avenues for more efficient and sustainable IoT deployments, reducing the environmental impact of battery disposal. With each synchronised connection established by CardioSync, we move closer to realising a more connected and sustainable world.