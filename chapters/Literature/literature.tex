\chapter{Literature and Related work}
This chapter delves into an in-depth exploration of existing research and developments in the field of batteryless IoT devices, intermittent connectivity solutions, and the pivotal role of the FreeBie architecture. By examining advancements, challenges, and critical gaps in the literature, this section lays the foundation for understanding the significance of the proposed CardioSync framework. The subsequent sections discuss key areas of interest, shedding light on the evolving landscape of batteryless IoT and intermittent communication in the context of modern technological demands.

\section{Advancements in Battery-Free IoT for Wireless Sensor Networks}
The field of batteryless IoT has garnered significant attention, driven by the promise of sustainable, autonomous operation. This approach offers extended device lifetimes and reduced environmental impact. Researchers and innovators have actively explored diverse energy harvesting techniques to power these devices, enabling applications across various domains such as agriculture, logistics, and environmental monitoring. Early works like the development of wireless sensor networks using simple solar energy harvesters and cost-effective energy storage units \cite{4394148} laid the groundwork for the subsequent advancements.

\noindent A notable milestone was achieved in 2020 with the introduction of a system utilizing ambient RF energy to power batteryless tags \cite{10.1145/3386901.3396604}. This breakthrough highlighted the potential of harnessing ubiquitous energy sources for practical applications. Intriguingly, the employment of a piezoelectric converter as an energy harvester to power a Bluetooth board through low-voltage vibration electromagnetic conversion was explored \cite{9221051}. This exploration validates the spectrum of solutions available to enhance the batteryless capability of IoT devices while still facilitating the formation of sensor networks. In 2017, the integration of solar energy harvesting chips and super capacitors in a batteryless sensor tag \cite{7990978} showcased the successful integration of multiple sensors, including temperature, humidity, and gas sensors, with efficient data transmission through BLE communication. This advancement proved particularly promising for industrial applications, hinting at the viability of batteryless IoT in various sectors.

\noindent A distinct stride was taken towards achieving batteryless communication through the successful design and testing of a wireless LoRaWAN end sensor node \cite{9299539}. This innovative approach demonstrates the feasibility of batteryless IoT even in long-range communication scenarios, further expanding the scope of its potential applications.

\noindent These strides exemplify only a subset of the numerous breakthroughs within the domain of battery-free IoT wireless communication. The works \cite{9718062}, \cite{10101211}, \cite{10.1145/3276774.3282823} offer additional evidence of the expanding landscape of battery-free IoT. Collectively, these advancements illuminate the dynamic landscape of battery-free IoT, highlighting its capacity to revolutionise myriad domains, from conventional industries to cutting-edge technologies.

\subsection{The FreeBie}
In the realm of intermittent connectivity solutions, the FreeBie architecture emerges as a pivotal contender, offering a distinctive approach to achieving Bluetooth Low Energy (BLE) communication on intermittently-powered wireless devices \cite{de2022Intermittently}. This architecture introduces an adaptive framework that tailors connection parameters according to the available harvested energy, facilitating efficient communication in resource-constrained environments. One of its significant features is supporting preemptive scheduling, allowing network processes to take precedence over application or operating system processes.

\noindent The proposed FreeBie architecture showcases impressive performance in maintaining wireless protocol communication on intermittently-powered devices. Its unique ability to enable bi-directional communication and dynamically manage network connections fills a critical gap in the domain of intermittently-powered devices.

\noindent However, the architecture relies on external components such as Ferroelectric Random Access Memory (FRAM) and a Real-Time Clock (RTC), impacting factors like system cost, size, and energy consumption. To address this, the authors suggest future exploration into developing a FreeBie version that integrates next-generation System on Chip (SoC) technology and leverages more energy-efficient harvesters.

\noindent Furthermore, the FreeBie architecture is designed to support intermittently-powered end devices, but a notable research gap lies in the absence of support for intermittently-powered hosts on both sides of BLE communication. While the architecture excels in enabling communication between intermittently-powered device and continuously powered hosts, there's potential for innovation in extending its capabilities to encompass two intermittently-powered end nodes. This extension could unlock novel use cases in wireless sensor networks, broadening the applicability of the FreeBie architecture.


\section{Body Sensor Network}
The evolution of Body Sensor Networks (BSNs) has marked a significant milestone in healthcare and wellness monitoring. These networks, composed of wearable sensors, offer real-time data collection, analysis, and transmission, empowering individuals and healthcare professionals with valuable insights into physiological and medical conditions. BSNs have demonstrated remarkable potential in applications ranging from remote patient monitoring to sports performance analysis and beyond.

\noindent Early research in BSNs primarily focused on sensor integration and data aggregation techniques \cite{5678072}. These studies paved the way for more advanced BSNs capable of real-time health monitoring. The advent of wearable devices with integrated physiological sensors has led to the development of innovative solutions for continuous monitoring of vital signs such as heart rate, temperature, and electrocardiogram (ECG) signals \cite{BSNreview}, \cite{6555588}. These advancements enable early detection of anomalies and timely intervention in critical situations.

\noindent The energy requirements of BSNs vary based on the complexity of sensors and the data transmission frequency. Wearable sensors that capture high-resolution data, such as ECG signals, demand a continuous power source for accurate monitoring. However, battery limitations hinder the potential for uninterrupted data collection. This challenge becomes even more pronounced when considering the size and weight restrictions of wearable devices \cite{WANG2020112410}, \cite{4755157}, \cite{6755575}.

\noindent The integration of battery-free IoT devices within BSNs offers a promising solution to the energy challenge \cite{5370806}. This innovation not only extends the operational lifetime of BSNs but also opens the door to continuous and sustainable monitoring. Also it could be affordable (less than US\$2 each when manufactured in volume), disposable, small, and easy to use \cite{5370806}. Also there are already some Battery free Wireless BSN solutions but they do use near-field-enabled clothing capable of establishing wireless power and data connectivity between multiple distant points around the body to create a network of battery-free sensors interconnected by proximity to functional textile patterns.\cite{Lin_Kim_Achavananthadith_Kurt_Tan_Yao_Tee_Lee_Ho_2020}

\noindent In summary, the evolution of BSNs has been characterised by breakthroughs in sensor integration and wireless communication. However, the challenge of battery dependence has persisted. The emergence of battery-free IoT devices powered by energy harvesting techniques presents a transformative solution. As ongoing research continues to address energy challenges and optimise device performance, the future of BSNs holds the promise of continuous and sustainable healthcare monitoring.